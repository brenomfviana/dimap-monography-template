\usepackage[brazil,hyperpageref]{backref}
\usepackage[alf,bibjustif]{abntex2cite}
\usepackage{indentfirst}
\usepackage{microtype}

\usepackage{mathptmx}
\usepackage{url}
\usepackage{dsfont}
\usepackage{lmodern}
\usepackage{ragged2e}
\usepackage{amssymb}
\usepackage{amsmath}
\usepackage{multirow}
\usepackage{colortbl}
\usepackage{xcolor}
\usepackage{graphicx}
\usepackage{booktabs}
\usepackage{pdflscape}
% \usepackage{changepage}
\usepackage{caption}
\usepackage{subcaption}

\usepackage{geometry}
\newgeometry{left=3cm,right=2cm,top=3cm,bottom=2cm}

%%% UFRN
% Padrão dos algoritmos
\usepackage{algorithm}
% \usepackage{algorithmic}
% \usepackage{algorithmicx}
\usepackage{algpseudocode}
%
\usepackage{xpatch}
\makeatletter
\xpatchcmd{\algorithmic}{\itemsep\z@}{\itemsep=-4pt}{}{}
\makeatother
%
\makeatletter
\newenvironment{breakablealgorithm}
{% \begin{breakablealgorithm}
\begin{center}
 \refstepcounter{algorithm}% New algorithm
 \hrule height.8pt depth0pt \kern2pt% \@fs@pre for \@fs@ruled
 \renewcommand{\caption}[2][\relax]{% Make a new \caption
   {\raggedright\textbf{\ALG@name~\thealgorithm} ##2\par}%
   \ifx\relax##1\relax % #1 is \relax
     \addcontentsline{loa}{algorithm}{\protect\numberline{\thealgorithm}##2}%
   \else % #1 is not \relax
     \addcontentsline{loa}{algorithm}{\protect\numberline{\thealgorithm}##1}%
   \fi
   \kern2pt\hrule\kern2pt
 }
}{% \end{breakablealgorithm}
 \kern2pt\hrule\relax% \@fs@post for \@fs@ruled
\end{center}
}
\renewcommand*{\ALG@name}{Algoritmo}
\makeatother
%
\usepackage{setspace}
\let\Algorithm\algorithm
\renewcommand\algorithm[1][]{\Algorithm[#1]\setstretch{0.5}}
%
\renewcommand{\listalgorithmname}{Lista de algoritmos}
\let\oldlistofalgorithms\listofalgorithms
\let\oldnumberline\numberline%
\newcommand{\algnumberline}[1]{Algoritmo~#1 -- }
\renewcommand{\listofalgorithms}{%
  \let\numberline\algnumberline%
  \oldlistofalgorithms
  \let\numberline\oldnumberline%
}
\usepackage{parskip}
\usepackage{etoolbox}
\makeatletter
\patchcmd{\@chapter}
  {\addtocontents{loa}}
  {\addtocontents{loa}{\protect\addvspace{0pt}}}
  {}{}
\makeatother
% Declaracoes em Português
\algrenewcommand\algorithmicend{\textbf{fim}}
\algrenewcommand\algorithmicdo{\textbf{faça}}
\algrenewcommand\algorithmicwhile{\textbf{enquanto}}
\algrenewcommand\algorithmicfor{\textbf{para}}
\algrenewcommand\algorithmicforall{\textbf{para cada}}
\algrenewcommand\algorithmicif{\textbf{se}}
\algrenewcommand\algorithmicthen{\textbf{então}}
\algrenewcommand\algorithmicelse{\textbf{senão}}
\algrenewcommand\algorithmicreturn{\textbf{devolve}}
\algrenewcommand\algorithmicfunction{\textbf{função}}
\algrenewcommand\algorithmicprocedure{\textbf{procedimento}}
\algrenewcommand\algorithmicreturn{\textbf{retorna}}

% Rearranja os finais de cada estrutura
\algrenewtext{EndWhile}{\algorithmicend\ \algorithmicwhile}
\algrenewtext{EndFor}{\algorithmicend\ \algorithmicfor}
\algrenewtext{EndIf}{\algorithmicend\ \algorithmicif}
\algrenewtext{EndFunction}{\algorithmicend\ \algorithmicfunction}

% O comando For, a seguir, retorna 'para #1 -- #2 até #3 faça'
\algnewcommand\algorithmicto{\textbf{até}}
\algrenewtext{For}[3]%
{\algorithmicfor\ #1 $\gets$ #2 \algorithmicto\ #3 \algorithmicdo}


% Lista de abreviaturas
\usepackage{acro}
\DeclareAcronym{ufrn}{
  short = UFRN,
  long = Universidade Federal do Rio Grande do Norte,
  class = abbrev
}

\DeclareAcronym{dimap}{
  short = DIMAp,
  long = Departamento de Informática e Matemática Aplicada,
  class = abbrev
}

\newlist{acronyms}{description}{1}
\newcommand*\addcolon[1]{\normalfont #1 --}
\setlist[acronyms]{labelwidth=0em,leftmargin =0em,noitemsep,itemindent=0pt,font=\addcolon}
\DeclareAcroListStyle{dimap}{list}{list=acronyms}
\acsetup{list-style=dimap}
\acsetup{first-style=short}
\acsetup{hyperref=true}
\acsetup{page-name=Abreviações}
% Lista de símbolos
\DeclareAcronym{lambda}{
  short = $\lambda$,
  long = (Algum símbolo),
  class = symbols
}

%%% Correções
% Sumário
\addtolength{\cftlastnumwidth}{-2em} % Editar de acordo com o indice de maior enumeração
\cftsetindents{part}{0em}{\cftlastnumwidth}
\cftsetindents{chapter}{0em}{\cftlastnumwidth}
\cftsetindents{section}{0em}{\cftlastnumwidth}
\cftsetindents{subsection}{0em}{\cftlastnumwidth}
\cftsetindents{subsubsection}{0em}{\cftlastnumwidth}
\cftsetindents{paragraph}{0em}{\cftlastnumwidth}
\cftsetindents{subparagraph}{0em}{\cftlastnumwidth}
\renewcommand{\cftsubsectionfont}{\normalfont\normalsize}
\renewcommand{\cftsubsubsectionfont}{\normalfont\small}
% Espaçamento do paragráfo
\setlength{\parindent}{1.25cm}
% Fonte da legenda
\captionsetup{font=sf}
% Fonte dos capítulos
\renewcommand{\ABNTEXchapterfont}{\rmfamily\bfseries\selectfont}
% URL links
\hypersetup{pageanchor=false, colorlinks=true, linkcolor=black, citecolor=black, urlcolor=black}
% Cabeçalhos
\renewcommand{\textual}{\pagestyle{abntchapfirst}\aliaspagestyle{chapter}{abntchapfirst}}
% Autoref
\renewcommand{\partautorefname}{Parte}
\renewcommand{\appendixautorefname}{Apêndice}
\renewcommand{\chapterautorefname}{Capítulo}
\renewcommand{\sectionautorefname}{\uppercase{S}eção}
\renewcommand{\subsectionautorefname}{\uppercase{S}ubseção}
\renewcommand{\subsubsectionautorefname}{\uppercase{S}ubsubseção}
\renewcommand{\figureautorefname}{Figura}
\renewcommand{\tableautorefname}{Tabela}
\renewcommand{\equationautorefname}{Equação}
\newcommand{\algorithmautorefname}{Algoritmo}
\renewcommand{\FancyVerbLineautorefname}{Linha}
\renewcommand{\theoremautorefname}{Teorema}
\renewcommand{\pageautorefname}{Página}
\addto\extrasbrazil{
  \def\sectionautorefname{Seção}
  \def\subsectionautorefname{Subseção}
  \def\subsubsectionautorefname{Subsubseção}
}
% Fix counter
\usepackage{chngcntr}
\counterwithin{figure}{chapter}
\counterwithin{table}{chapter}
\counterwithin{equation}{chapter}
\counterwithin{algorithm}{chapter}
% Remove warnings
\pdfstringdefDisableCommands{\let\uppercase\relax}
%%%
