% Papel: A4 – cor branca
% Fonte: Times New Roman ou Arial- tamanho 12 – cor: preta. Nas citações com mais de 3 linhas, notas de rodapé, legendas e tabelas a fonte deve ter o tamanho 10.
% Itálico: Deve ser usado nas palavras de outros idiomas. Esta orientação não se aplica às expressões latinas apud e et al.
% Margens: Direita e inferior: 2cm / Esquerda e superior: 3cm
% Parágrafos / Espaçamento: 1,5 entre linhas;

\documentclass[a4paper,12pt,openany,oneside]{abntex2}
\usepackage[brazil]{babel}
\usepackage[utf8]{inputenc}
\usepackage[T1]{fontenc}

\usepackage[brazil,hyperpageref]{backref}
\usepackage[alf,bibjustif]{abntex2cite}
\usepackage{indentfirst}
\usepackage{microtype}

\usepackage{mathptmx}
\usepackage{url}
\usepackage{dsfont}
\usepackage{lmodern}
\usepackage{ragged2e}
\usepackage{amssymb}
\usepackage{amsmath}
\usepackage{multirow}
\usepackage{colortbl}
\usepackage{xcolor}
\usepackage{graphicx}
\usepackage{booktabs}
\usepackage{pdflscape}
% \usepackage{changepage}
\usepackage{caption}
\usepackage{subcaption}

\usepackage{geometry}
\newgeometry{left=3cm,right=2cm,top=3cm,bottom=2cm}

%%% UFRN
% Padrão dos algoritmos
\usepackage{algorithm}
% \usepackage{algorithmic}
% \usepackage{algorithmicx}
\usepackage{algpseudocode}
%
\usepackage{xpatch}
\makeatletter
\xpatchcmd{\algorithmic}{\itemsep\z@}{\itemsep=-4pt}{}{}
\makeatother
%
\makeatletter
\newenvironment{breakablealgorithm}
{% \begin{breakablealgorithm}
\begin{center}
 \refstepcounter{algorithm}% New algorithm
 \hrule height.8pt depth0pt \kern2pt% \@fs@pre for \@fs@ruled
 \renewcommand{\caption}[2][\relax]{% Make a new \caption
   {\raggedright\textbf{\ALG@name~\thealgorithm} ##2\par}%
   \ifx\relax##1\relax % #1 is \relax
     \addcontentsline{loa}{algorithm}{\protect\numberline{\thealgorithm}##2}%
   \else % #1 is not \relax
     \addcontentsline{loa}{algorithm}{\protect\numberline{\thealgorithm}##1}%
   \fi
   \kern2pt\hrule\kern2pt
 }
}{% \end{breakablealgorithm}
 \kern2pt\hrule\relax% \@fs@post for \@fs@ruled
\end{center}
}
\renewcommand*{\ALG@name}{Algoritmo}
\makeatother
%
\usepackage{setspace}
\let\Algorithm\algorithm
\renewcommand\algorithm[1][]{\Algorithm[#1]\setstretch{0.5}}
%
\renewcommand{\listalgorithmname}{Lista de algoritmos}
\let\oldlistofalgorithms\listofalgorithms
\let\oldnumberline\numberline%
\newcommand{\algnumberline}[1]{Algoritmo~#1 -- }
\renewcommand{\listofalgorithms}{%
  \let\numberline\algnumberline%
  \oldlistofalgorithms
  \let\numberline\oldnumberline%
}
\usepackage{parskip}
\usepackage{etoolbox}
\makeatletter
\patchcmd{\@chapter}
  {\addtocontents{loa}}
  {\addtocontents{loa}{\protect\addvspace{0pt}}}
  {}{}
\makeatother
% Declaracoes em Português
\algrenewcommand\algorithmicend{\textbf{fim}}
\algrenewcommand\algorithmicdo{\textbf{faça}}
\algrenewcommand\algorithmicwhile{\textbf{enquanto}}
\algrenewcommand\algorithmicfor{\textbf{para}}
\algrenewcommand\algorithmicforall{\textbf{para cada}}
\algrenewcommand\algorithmicif{\textbf{se}}
\algrenewcommand\algorithmicthen{\textbf{então}}
\algrenewcommand\algorithmicelse{\textbf{senão}}
\algrenewcommand\algorithmicreturn{\textbf{devolve}}
\algrenewcommand\algorithmicfunction{\textbf{função}}
\algrenewcommand\algorithmicprocedure{\textbf{procedimento}}
\algrenewcommand\algorithmicreturn{\textbf{retorna}}

% Rearranja os finais de cada estrutura
\algrenewtext{EndWhile}{\algorithmicend\ \algorithmicwhile}
\algrenewtext{EndFor}{\algorithmicend\ \algorithmicfor}
\algrenewtext{EndIf}{\algorithmicend\ \algorithmicif}
\algrenewtext{EndFunction}{\algorithmicend\ \algorithmicfunction}

% O comando For, a seguir, retorna 'para #1 -- #2 até #3 faça'
\algnewcommand\algorithmicto{\textbf{até}}
\algrenewtext{For}[3]%
{\algorithmicfor\ #1 $\gets$ #2 \algorithmicto\ #3 \algorithmicdo}


% Lista de abreviaturas
\usepackage{acro}
\DeclareAcronym{ufrn}{
  short = UFRN,
  long = Universidade Federal do Rio Grande do Norte,
  class = abbrev
}

\DeclareAcronym{dimap}{
  short = DIMAp,
  long = Departamento de Informática e Matemática Aplicada,
  class = abbrev
}

\newlist{acronyms}{description}{1}
\newcommand*\addcolon[1]{\normalfont #1 --}
\setlist[acronyms]{labelwidth=0em,leftmargin =0em,noitemsep,itemindent=0pt,font=\addcolon}
\DeclareAcroListStyle{dimap}{list}{list=acronyms}
\acsetup{list-style=dimap}
\acsetup{first-style=short}
\acsetup{hyperref=true}
\acsetup{page-name=Abreviações}
% Lista de símbolos
\DeclareAcronym{lambda}{
  short = $\lambda$,
  long = (Algum símbolo),
  class = symbols
}

%%% Correções
% Sumário
\addtolength{\cftlastnumwidth}{-2em} % Editar de acordo com o indice de maior enumeração
\cftsetindents{part}{0em}{\cftlastnumwidth}
\cftsetindents{chapter}{0em}{\cftlastnumwidth}
\cftsetindents{section}{0em}{\cftlastnumwidth}
\cftsetindents{subsection}{0em}{\cftlastnumwidth}
\cftsetindents{subsubsection}{0em}{\cftlastnumwidth}
\cftsetindents{paragraph}{0em}{\cftlastnumwidth}
\cftsetindents{subparagraph}{0em}{\cftlastnumwidth}
\renewcommand{\cftsubsectionfont}{\normalfont\normalsize}
\renewcommand{\cftsubsubsectionfont}{\normalfont\small}
% Espaçamento do paragráfo
\setlength{\parindent}{1.25cm}
% Fonte da legenda
\captionsetup{font=sf}
% Fonte dos capítulos
\renewcommand{\ABNTEXchapterfont}{\rmfamily\bfseries\selectfont}
% URL links
\hypersetup{pageanchor=false, colorlinks=true, linkcolor=black, citecolor=black, urlcolor=black}
% Cabeçalhos
\renewcommand{\textual}{\pagestyle{abntchapfirst}\aliaspagestyle{chapter}{abntchapfirst}}
% Autoref
\renewcommand{\partautorefname}{Parte}
\renewcommand{\appendixautorefname}{Apêndice}
\renewcommand{\chapterautorefname}{Capítulo}
\renewcommand{\sectionautorefname}{\uppercase{S}eção}
\renewcommand{\subsectionautorefname}{\uppercase{S}ubseção}
\renewcommand{\subsubsectionautorefname}{\uppercase{S}ubsubseção}
\renewcommand{\figureautorefname}{Figura}
\renewcommand{\tableautorefname}{Tabela}
\renewcommand{\equationautorefname}{Equação}
\newcommand{\algorithmautorefname}{Algoritmo}
\renewcommand{\FancyVerbLineautorefname}{Linha}
\renewcommand{\theoremautorefname}{Teorema}
\renewcommand{\pageautorefname}{Página}
\addto\extrasbrazil{
  \def\sectionautorefname{Seção}
  \def\subsectionautorefname{Subseção}
  \def\subsubsectionautorefname{Subsubseção}
}
% Fix counter
\usepackage{chngcntr}
\counterwithin{figure}{chapter}
\counterwithin{table}{chapter}
\counterwithin{equation}{chapter}
\counterwithin{algorithm}{chapter}
% Remove warnings
\pdfstringdefDisableCommands{\let\uppercase\relax}
%%%

\usepackage{titling}
\title{Título do Trabalho}
\author{Nome do(a) author(a)}
\newcommand{\fltitle}{Título do trabalho (em língua estrangeira)}
\newcommand{\advisor}{Titulação e nome do(a) orientador(a)}
\newcommand{\coadvisor}{Titulação e nome do(a) co-orientador(a)}
\renewcommand{\local}{Natal - RN}
\renewcommand{\date}{Data}
\newcommand{\approvaldate}{Data}


\begin{document}

\frontmatter % = \pretextual
\frenchspacing

\pagenumbering{roman}

\begin{center}

\begin{minipage}{2cm}
    \begin{center}
    \includegraphics[width=1.7cm, height=2.0cm]{frontmatter/images/brasao-ufrn.png}
    \end{center}
\end{minipage}
\begin{minipage}{11cm}
    \begin{center}
        \begin{SingleSpace}
        \textsc{ \small
        Universidade Federal do Rio Grande do Norte\\
        Centro de Ciências Exatas e da Terra\\
        Departamento de Informática e Matemática Aplicada\\
        Bacharelado em Ciência da Computação }
        \end{SingleSpace}
    \end{center}
\end{minipage}
\begin{minipage}{2cm}
    \begin{center}
    \includegraphics[width=1.8cm, height=1.5cm]{frontmatter/images/logo-dimap.png}
    \end{center}
\end{minipage}

\vfill

{\setlength{\baselineskip} {1.3\baselineskip}
{\LARGE \textbf{\thetitle}}\par}

\vfill

{\large \textbf{\theauthor}}

\vfill

\local \\ \date

\end{center}
\begin{center}

\textbf{\large \theauthor}

\vfill

\textbf{\Large \thetitle}

\vfill

\begin{SingleSpace}
\begin{adjustwidth}{.51\textwidth}{0cm}
\noindent \justify Monografia de Graduação apresentada ao Departamento de
Informática e Matemática Aplicada do Centro de Ciências Exatas e da Terra da
Universidade Federal do Rio Grande do Norte como requisito parcial para a obtenção
do grau de bacharel em Ciência da Computação.
\end{adjustwidth}
\end{SingleSpace}

\vfill

Orientador(a)\\ \advisor

\vspace{0.8cm}

Co-orientador(a)\\ \coadvisor

\vfill

\begin{SingleSpace}
\textsc{Universidade Federal do Rio Grande do Norte -- UFRN\\
Departamento de Informática e Matemática Aplicada -- DIMAp }
\end{SingleSpace}

\vfill

\local \\ \date

\end{center}
\begin{folhadeaprovacao}
    \noindent Monografia de Graduação sob o título \textit{\thetitle} apresentada por \theauthor e aceita pelo Departamento de Informática e Matemática Aplicada do Centro de Ciências Exatas e da Terra da Universidade Federal do Rio Grande do Norte, sendo aprovada por todos os membros da banca examinadora abaixo especificada:
    
    \vfill
    
    \setlength{\ABNTEXsignthickness}{0.4pt}
    \setlength{\ABNTEXsignwidth}{10cm}
    \setlength{\ABNTEXsignskip}{2.5cm}
    
    \assinatura{\advisor\\
    {\small Orientador(a)} \\
    {\footnotesize
    Departamento\\
    Universidade}
    }
    
    \assinatura{\coadvisor \\
    {\small Co-orientador(a)} \\
    {\footnotesize
    Departamento\\
    Universidade}
    }
    
    \assinatura{Titulação e nome do membro da banca examinadora \\
    {\footnotesize
    Departamento\\
    Universidade}
    }
    
    \assinatura{Titulação e nome do membro da banca examinadora \\ 
    {\footnotesize
    Departamento \\
    Universidade}
    }
    
    \vfill
    
    \begin{center}
    \local, \approvaldate.
    \end{center}
\end{folhadeaprovacao}


\begin{dedicatoria}

~\vfill

\begin{flushright}
À minha família e amigos.
\end{flushright}

\vspace{5cm}~

\end{dedicatoria}
\include{frontmatter/agradecimentos}
\begin{epigrafe}

~\vfill

\begin{flushright}
\textit{Citação}\medskip\\
Autor
\end{flushright}

\vspace{5cm}~

\end{epigrafe}


\renewcommand{\thepage}{\roman{page}}
\setcounter{page}{1}

\include{frontmatter/resumo}
\include{frontmatter/abstract}

% Lista de figuras
\pdfbookmark[0]{\listfigurename}{lof}
\listoffigures
\cleardoublepage
% Lista de tabelas
\pdfbookmark[0]{\listtablename}{lot}
\listoftables
\cleardoublepage
% Lista de algoritmos
\listofalgorithms
\addcontentsline{toc}{chapter}{Lista de algoritmos}
\cleardoublepage
% Lista de abreviaturas
\printacronyms[include-classes=abbrev,heading=chapter*,name=Lista de abreviaturas e siglas]
\cleardoublepage
% Lista de símbolos
\printacronyms[include-classes=symbols,heading=chapter*,name=Lista de símbolos]
\cleardoublepage
% Sumário
\pdfbookmark[0]{\contentsname}{toc}
\tableofcontents
\cleardoublepage

\mainmatter % = \textual
\renewcommand{\thepage}{\arabic{page}}
\setcounter{page}{1}

\pagenumbering{arabic}

\include{mainmatter/intro}
\include{mainmatter/cap2}
% Capítulo 3
\chapter{Capítulo 3}

Algumas regras devem ser observadas na redação da monografia:
\begin{enumerate}
	\item ser claro, preciso, direto, objetivo e conciso, utilizando frases curtas e evitando ordens inversas desnecessárias;
	\item construir períodos com no máximo duas ou três linhas, bem como parágrafos com cinco linhas cheias, em média, e no máximo oito (ou seja, não construir parágrafos e períodos muito longos, pois isso cansa o(s) leitor(es) e pode fazer com que ele(s) percam a linha de raciocínio desenvolvida);
	\item a simplicidade deve ser condição essencial do texto; a simplicidade do texto não implica necessariamente repetição de formas e frases desgastadas, uso exagerado de voz passiva (como \textit{será iniciado}, \textit{será realizado}), pobreza vocabular etc. Com palavras conhecidas de todos, é possível escrever de maneira original e criativa e produzir frases elegantes, variadas, fluentes e bem alinhavadas;
	\item adotar como norma a ordem direta, por ser aquela que conduz mais facilmente o leitor à essência do texto, dispensando detalhes irrelevantes e indo diretamente ao que interessa, sem rodeios (verborragias);
	\item não começar períodos ou parágrafos seguidos com a mesma palavra, nem usar repetidamente a mesma estrutura de frase;
	\item desprezar as longas descrições e relatar o fato no menor número possível de palavras;
	\item recorrer aos termos técnicos somente quando absolutamente indispensáveis e nesse caso colocar o seu significado entre parênteses (ou seja, não se deve admitir que todos os que lerão o trabalho já dispõem de algum conhecimento desenvolvido no mesmo);
	\item dispensar palavras e formas empoladas ou rebuscadas, que tentem transmitir ao leitor mera idéia de erudição;
	\item não perder de vista o universo vocabular do leitor, adotando a seguinte regra prática: \textit{nunca escrever o que não se diria};
	\item termos coloquiais ou de gíria devem ser usados com \textit{extrema} parcimônia (ou mesmo nem serem utilizados) e apenas em casos muito especiais, para não darem ao leitor a idéia de vulgaridade e descaracterizar o trabalho;
	\item ser rigoroso na escolha das palavras do texto, desconfiando dos sinônimos perfeitos ou de termos que sirvam para todas as ocasiões; em geral, há uma palavra para definir uma situação;
	\item encadear o assunto de maneira suave e harmoniosa, evitando a criação de um texto onde os parágrafos se sucedem uns aos outros como compartimentos estanques, sem nenhuma fluência entre si;
	\item ter um extremo cuidado durante a redação do texto, principalmente com relação às regras gramaticais e ortográficas da língua; geralmente todo o texto é escrito na forma impessoal do verbo, não se utilizando, portanto, de termos em primeira pessoa, seja do plural ou do singular.
\end{enumerate}


\section{Seção 1}

Teste de uma tabela:

\begin{table}[htb]
% Título de tabelas sempre aparecem antes da tabela
\caption{Tabela sem sentido}
\label{tab:TabelaSemSentido}
\center
{
	\begin{tabular}{l|l}
		\hline
		Titulo Coluna 1   & Título Coluna 2\\
		\hline
		X                 & Y\\
		X                 & W\\
		\hline
	\end{tabular}
}
\end{table}


\section{Seção 2}

Seção 2


\subsection{Subseção 2.1}

Referência à tabela definida no início: \ref{tab:TabelaSemSentido}


\subsection{Subseção 2.2}

Subsection 2.2


\section{Seção 3}

Seção 3
% Capítulo 4
\chapter{Capítulo 4}

\section{Seção 1}

Teste para símbolo

\ac{lambda}


\section{Seção 2}

Teste para abreviatura 

\ac{ufrn}

\ac{dimap}


\begin{breakablealgorithm}
\begin{algorithmic}[1]
\If {$i\geq maxval$}
    \State $i\gets 0$
\Else
    \If {$i+k\leq maxval$}
        \State $i\gets i+k$
    \EndIf
\EndIf
\end{algorithmic}
\caption{Esperança}
\label{alg1}
\end{breakablealgorithm}
\include{mainmatter/cap5}
\include{mainmatter/conclusao}

\bibliography{mainmatter/refs}
\bibliographystyle{abntex2-alf}
\bibliographystyle{abntex2-num}


\backmatter % = \postextual

\begin{apendicesenv}
\partapendices
\include{backmatter/apendicea}
\end{apendicesenv}

\begin{anexosenv}
\partanexos
% \annex
\chapter{Primeiro Anexo}

Os anexos são textos ou documentos não elaborado pelo autor, que servem de fundamentação, comprovação e ilustração.
\end{anexosenv}

\end{document}
