\begin{center}
\Large \textbf{\fltitle}
\end{center}

\vspace{1cm}

\begin{flushright}
Author: Nome do(a) aluno(a)\\
Advisor: \advisor\\
Co-advisor: \coadvisor
\end{flushright}

\vspace{1cm}

\begin{center}
	\Large{\textsc{\textbf{Abstract}}}
\end{center}

\noindent O resumo em língua estrangeira (em inglês \textit{Abstract}, em espanhol \textit{Resumen}, em francês \textit{Résumé}) é uma versão do resumo escrito na língua vernácula para idioma de divulgação internacional. Ele deve apresentar as mesmas características do anterior (incluindo as mesmas palavras, isto é, seu conteúdo não deve diferir do resumo anterior), bem como ser seguido das palavras representativas do conteúdo do trabalho, isto é, palavras-chave e/ou descritores, na língua estrangeira. Embora a especificação abaixo considere o inglês como língua estrangeira (o mais comum), não fica impedido a adoção de outras linguas (a exemplo de espanhol ou francês) para redação do resumo em língua estrangeira.
\vspace{\onelineskip}

\noindent\textit{Keywords}: Keyword 1, Keyword 2, Keyword 3.